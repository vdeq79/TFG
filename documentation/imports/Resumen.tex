\chapter*{Resumen}
En ocasiones, se necesita estudiar sucesos que no son directamente observables pero que generan una serie de consecuencias apreciables. Para inferir sobre los sucesos ocultos podemos utilizar los modelos de Markov ocultos, un modelo estadístico que es capaz de proporcionarnos información de eventos no observables a partir de unas salidas conocidas. 

En este trabajo se estudian los elementos que constituyen un modelo de Markov oculto y los algoritmos derivados de dichos modelos. Algunos de los cuales, basándose en la  programación dinámica. Con dichos algoritmos, podemos inferir sobre los eventos ocultos a partir de una sucesión de observaciones. Previo al estudio de los modelos de Markov ocultos, introduciremos un tipo de proceso estocástico conocido como cadenas de Markov. El análisis de estos procesos es fundamental, ya que los modelos ocultos pueden considerarse como una generalización de cadenas de Markov. Finalmente, para mostrar las aplicaciones de los modelos de Markov ocultos en la biología, presentaremos e implementaremos modelos que se utilizan en problemas específicos, en concreto, para alineamiento y análisis de secuencias.

\textbf{Palabras clave:} Probabilidad, procesos estocásticos, cadenas de Markov, modelos de Markov ocultos, programación dinámica, bioinformática, secuencias biológicas.