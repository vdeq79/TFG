\chapter*{Conclusiones y vías futuras}
\addcontentsline{toc}{chapter}{Conclusiones y vías futuras}

En este trabajo se ha realizado un estudio exhaustivo de los modelos de Markov ocultos y sus aplicaciones, centrándose especialmente en el ámbito de la biología. A partir de los conocimientos de probabilidad y de matrices positivas, se ha analizado con detalle el comportamiento de las cadenas de Markov, necesarias para el estudio de los modelos ocultos. Una vez introducida la estructura y los elementos que conforman los HMMs, se han presentado los tres problemas asociados y los algoritmos que los resuelven. Se ha proporcionado un punto de vista matemático y probabilístico en la exposición de dichos algoritmos y se han añadido mejoras para evitar problemas en la implementación en ordenador. Se han explorado algunas de las aplicaciones de los HMMs en la biología y se han desarrollado las variantes correspondientes a dichas aplicaciones. En definitiva, se han logrado alcanzar los objetivos planteados desde el inicio del trabajo.

A lo largo del trabajo se ha podido observar la capacidad de los HMMs para adaptar a problemas concretos. Por ello, existen una gran cantidad de variantes como HMM generalizado (GHMM), HMM sensible de contexto (csHMM), etc. Como vías futuras se podrían explorar más variantes de HMMs. También sería interesante estudiar modelos cuyas emisiones siguen una distribución de probabilidad concreta, como por ejemplo Poisson HMM o modelos basados en una cadena de Markov con orden superior, esto es, una cadena de Markov que tiene en cuenta más estados pasados.