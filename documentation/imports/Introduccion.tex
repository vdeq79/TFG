\chapter*{Introducción y objetivos}
\addcontentsline{toc}{chapter}{Introducción y objetivos}

A lo largo de este trabajo se estudiará el modelo estadístico conocido como modelo de Markov oculto. Este modelo se puede considerar como una generalización de las cadenas de Markov, un tipo de proceso estocástico en el que la probabilidad de un evento determinado depende únicamente del evento inmediatamente anterior. Gracias a esta propiedad, las cadenas de Markov son capaces de modelar sistemas en los que el estado actual solo depende del estado anterior más reciente, en lugar de depender de todos los estados anteriores. Además, permiten la realización de predicciones: mediante el análisis de las transiciones entre los estados, podemos predecir el comportamiento futuro del sistema. 

Las cadenas de Markov tienen una amplia gama de aplicaciones. Se utilizan en la predicción del clima y las condiciones meteorológicas, el estudio de la evolución de las poblaciones en biología y epidemiología. También son útiles para modelar la evolución de activos financieros y juegos de azar. Además, se emplean en el algoritmo Pagerank de Google para determinar la posición de una página en los resultados de búsqueda.

Sin embargo, en muchas ocasiones los eventos que pretendemos estudiar no son directamente observables, lo que imposibilita el uso de cadenas de Markov en esos casos. Para abordar este problema, surgieron los modelos de Markov ocultos. Inicialmente fueron introducidos por Baum y Petrie en 1966 \cite{Baum} como funciones probabilísticas de cadenas de Markov, a partir de ahí se han desarrollado hasta llegar a la forma en que se conocen hoy en día. Mediante la utilización de estos modelos podemos estudiar sucesiones de procesos no observables conocidos como ocultos, a partir de una serie de observaciones relacionadas con los anteriores. Por esta razón, tienen una amplia aplicación en campos como el reconocimiento del habla, procesamiento de señales, reconocimiento de patrones, biología, epidemiología, etc. Entre todas estas posibles aplicaciones, nos centraremos en el uso de los modelos de Markov ocultos en la biología, en concreto, en la bioinformática. 

Se define la bioinformática como la ciencia que estudia y analiza grandes conjuntos de datos biológicos, y en particular, genéticos, mediante la aplicación de conocimientos matemáticos, estadísticos y de ciencias de la computación \cite{Warren}. El estudio de las secuencias genéticas nos ayuda a comprender los mecanismos biológicos en las células. Sin embargo, la gran cantidad de datos que supone hace que sea imposible extraer información relevante sin una herramienta eficaz. 

El uso de los modelos de Markov ocultos para este tipo de problema fue introducido por primera vez por Churchill en 1989 \cite{Churchill}. Anteriormente, ya habían sido empleados en el reconocimiento del habla en 1980 \cite{Ferguson} y destacaban por la eficacia a la hora de modelar correlaciones entre símbolos adyacentes, propiedades o eventos \cite{Rabiner}. 

Como habíamos mencionado anteriormente, el estudio de las secuencias genéticas nos proporciona información sobre sus funciones. Con frecuencia, se descubren nuevas secuencias. Para analizarlas, una de las posibilidades es comparar las similitudes significativas entre la nueva secuencia y una secuencia conocida. Dependiendo del resultado, podemos transferir información acerca de la estructura o la función a la nueva secuencia. Previo a la evaluación de similitudes, es necesario encontrar un alineamiento razonable entre las dos secuencias consideradas. 