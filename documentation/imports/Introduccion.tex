\chapter*{Introducción y objetivos}
\addcontentsline{toc}{chapter}{Introducción y objetivos}

En este trabajo se va a estudiar principalmente el proceso estocástico conocido como modelo de Markov oculto. Mediante la utilización de estos modelos podemos estudiar sucesiones de procesos no observables llamados ocultos a partir de una serie de consecuencias de los anteriores. Por ello, es ampliamente utilizado en el reconocimiento del habla, procesamiento de señales, reconocimiento de patrones, biología, epidemiología, etc. Entre todas estas posibles aplicaciones, nos centraremos en el uso de los modelos de Markov ocultos en la biología, en concreto, en la bioinformática. 

Se define la bioinformática como la ciencia que estudia y analiza grandes conjuntos de datos biológicos, y en particular, genéticos, mediante la aplicación de conocimientos matemáticos, estadísticos y de la ciencia de la computación \cite{Warren}. 

El estudio de las secuencias genéticas nos ayudan a comprender los mecanismos biológicos en las células, sin embargo, la gran cantidad de datos que supone hace que sea imposible extrae información relevante sin una herramienta eficaz. El uso de los modelos de Markov ocultos para este tipo de problema fue introducido por primera vez por Churchill en 1989 \cite{Churchill}. Anteriormente, ya habían sido utilizados en el reconocimiento del habla en 1980 \cite{Ferguson} y eran destacados por la eficacia a la hora de modelar correlaciones entre símbolos adyacentes, propiedades o eventos \cite{Yoon}. 