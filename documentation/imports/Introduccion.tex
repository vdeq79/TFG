\chapter*{Introducción y objetivos}
\addcontentsline{toc}{chapter}{Introducción y objetivos}

A lo largo de este trabajo se estudiará el modelo estadístico conocido como modelo de Markov oculto. Este modelo se puede considerar como una generalización de las cadenas de Markov, un tipo de proceso estocástico en el que la probabilidad de un evento determinado depende únicamente del evento inmediatamente anterior. Además, permiten la realización de predicciones: mediante el análisis de las transiciones entre los estados, podemos predecir el comportamiento futuro del sistema. 

Las cadenas de Markov tienen una amplia gama de aplicaciones. Se utilizan en la predicción del clima y las condiciones meteorológicas, el estudio de la evolución de las poblaciones en biología y epidemiología. También son útiles para modelar la evolución de activos financieros y juegos de azar. Además, se emplean en el algoritmo Pagerank de Google para determinar la posición de una página en los resultados de búsqueda.

Sin embargo, en muchas ocasiones los eventos que pretendemos estudiar no son directamente observables, lo que imposibilita el uso de cadenas de Markov en esos casos. Para abordar este problema, surgieron los modelos de Markov ocultos. Inicialmente fueron introducidos por Baum y Petrie en 1966 \cite{Baum} como funciones probabilísticas de cadenas de Markov. A partir de ahí se han desarrollado hasta llegar a la forma en que se conocen hoy en día. Mediante la utilización de estos modelos podemos estudiar sucesiones de procesos no observables conocidos como ocultos, a partir de una serie de observaciones relacionadas con los anteriores. Por esta razón, tienen una amplia aplicación en campos como el reconocimiento del habla, procesamiento de señales, reconocimiento de patrones, biología, epidemiología, etc. Entre todas estas posibles aplicaciones, nos centraremos en el uso de los modelos de Markov ocultos en la biología, en concreto, en la bioinformática. 

Se define la bioinformática como la ciencia que estudia y analiza grandes conjuntos de datos biológicos, y en particular, genéticos, mediante la aplicación de conocimientos matemáticos, estadísticos y de ciencias de la computación \cite{Warren}. El estudio de las secuencias genéticas nos ayuda a comprender los mecanismos biológicos en las células. Sin embargo, la gran cantidad de datos que supone hace que sea imposible extraer información relevante sin una herramienta eficaz. 

El uso de los modelos de Markov ocultos para este tipo de problema fue introducido por primera vez por Churchill en 1989 \cite{Churchill}. Anteriormente, ya habían sido empleados en el reconocimiento del habla en 1980 \cite{Ferguson} y destacaban por la eficacia a la hora de modelar correlaciones entre símbolos adyacentes, propiedades o eventos \cite{Rabiner}. 

El estudio de las secuencias genéticas nos proporciona información sobre sus funciones, con frecuencia, se descubren nuevas secuencias que requieren análisis. Una de las posibilidades es comparar las similitudes significativas entre la nueva secuencia y una secuencia conocida. Dependiendo del resultado, podemos inferir información acerca de la estructura o función de la nueva secuencia. Previo a la evaluación de similitudes, en general es necesario encontrar un alineamiento razonable entre las dos secuencias consideradas. Al hablar de alineamiento entre dos secuencias, tomamos en cuenta la posibilidad de insertar huecos en ambas secuencias. Supongamos que tenemos dos secuencias de ADN: 
\[x=TACGAACTG \qquad y=TCGTAACGTA\]
Un alineamiento razonable podría ser el siguiente:
\[\begin{array}{c c c c c c c c c c c}
    T& A& C& G& -& A& A& C& -& T& G \\
    T& -& C& G& T& A& A& C& G& T& A
\end{array}\]
podemos observar que, al insertar huecos en las secuencias, logramos alinear símbolos iguales en ambas secuencias. Para construir estos alineamientos, se diseña un modelo de puntuaciones a partir de consideraciones experimentales y estadísticas. Estas consideraciones también son aplicables en los modelos de Markov ocultos, lo que les permite alinear secuencias desde una perspectiva probabilística. Una ventaja significativa de los modelos de Markov ocultos es que nos permiten evaluar las similitudes entre dos secuencias independientemente de un alineamiento específico.

Esta es una de las posibles aplicaciones de los modelos de Markov ocultos en biología. Existen diversas variantes del modelo general adaptadas a problemas específicos. Esta capacidad de adaptación es una de las ventajas clave de los modelos de Markov ocultos, ya que permiten ajustar su estructura para simular situaciones particulares. Además, como veremos en este trabajo, también permiten el ajuste de sus parámetros basados en observaciones. Esto es de gran utilidad cuando desconocemos los estados que generan dichas observaciones, pues a partir de los parámetros podemos establecer relaciones entre los estados ocultos y las observaciones.

\subsection*{Descripción del trabajo}
El presente trabajo desarrolla la teoría de las cadenas de Markov y los modelos de Markov ocultos. Nos centraremos en los algoritmos utilizados en los modelos ocultos y los problemas que se pueden abordar a partir de dichos modelos. Desde el punto de vista matemático, se usarán principalmente conceptos relacionados con la probabilidad y la inferencia estadística. La programación dinámica será la herramienta informática fundamental en el trabajo. Para estudiar las aplicaciones de los modelos de Markov ocultos en la biología, implementaremos clases que simulen los modelos correspondientes en formato de Jupyter Notebook y los utilizaremos para ilustrar el funcionamiento de dichos modelos mediante ejemplos.    

\subsection*{Contenido de la memoria}
El trabajo se estructura en cuatro capítulos principales. En el primero, se recopilan conceptos básicos de probabilidad y programación dinámica. También presentaremos resultados necesarios sobre matrices positivas para el estudio de las cadenas de Markov. El segundo capítulo está dedicado a las cadenas de Markov. Se introducen definiciones y propiedades básicas que serán de gran utilidad más adelante para el desarrollo del siguiente capítulo. En concreto se hará una clasificación de los distintos estados de una cadena de Markov y se presentarán algunos resultados sobre el comportamiento y evolución del modelo a largo plazo. En el capítulo $3$ se introducen los conceptos básicos relacionados con los modelos de Markov ocultos y se explicarán detalladamente los algoritmos asociados a ellos. Finalmente, analizaremos problemas de biología que pueden ser abordados mediante el uso de los modelos ocultos o sus variantes.


\subsection*{Objetivos del trabajo}
Los objetivos que se pretenden abordar en este trabajo son:
\begin{itemize}
    \item Estudiar las cadenas de Markov, los tipos de estados y su aplicación para inferir comportamiento a largo plazo.
    \item Estudiar los modelos de Markov ocultos, así como los algoritmos relacionados.
    \item Estudiar las aplicaciones de los modelos de Markov ocultos en la biología, analizar variantes adaptadas a problemas particulares e implementar de dichos variantes.
\end{itemize}