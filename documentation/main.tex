% Plantilla para un Trabajo Fin de Grado de la Universidad de Granada,
% adaptada para el Doble Grado en Ingeniería Informática y Matemáticas.
%
%  Autor: Mario Román.
%  Licencia: GNU GPLv2.
%
% Esta plantilla es una adaptación al castellano de la plantilla
% classicthesis de André Miede, que puede obtenerse en:
%  https://ctan.org/tex-archive/macros/latex/contrib/classicthesis?lang=en
% La plantilla original se licencia en GNU GPLv2.
%
% Esta plantilla usa símbolos de la Universidad de Granada sujetos a la normativa
% de identidad visual corporativa, que puede encontrarse en:
% http://secretariageneral.ugr.es/pages/ivc/normativa
%
% La compilación se realiza con las siguientes instrucciones:
%   pdflatex --shell-escape main.tex
%   bibtex main
%   pdflatex --shell-escape main.tex
%   pdflatex --shell-escape main.tex

% Opciones del tipo de documento
\documentclass[oneside,openright,titlepage,numbers=noenddot,openany,headinclude,footinclude=true,
cleardoublepage=empty,abstractoff,BCOR=5mm,paper=a4,fontsize=12pt,main=spanish]{scrreprt}

% Paquetes de latex que se cargan al inicio. Cubren la entrada de
% texto, gráficos, código fuente y símbolos.
\usepackage[utf8]{inputenc}
\usepackage[T1]{fontenc}
\usepackage{fixltx2e}
\usepackage{graphicx} % Inclusión de imágenes.
\usepackage{grffile}  % Distintos formatos para imágenes.
\usepackage{longtable} % Tablas multipágina.
\usepackage{wrapfig} % Coloca texto alrededor de una figura.
\usepackage{rotating}
\usepackage[normalem]{ulem}
\usepackage{amsmath}
\usepackage{textcomp}
\usepackage{amssymb}
\usepackage{capt-of}
\usepackage[colorlinks=true]{hyperref}
\usepackage{tikz} % Diagramas conmutativos.
\usepackage{minted} % Código fuente.
\usepackage[T1]{fontenc}
\usepackage{natbib}


% Plantilla classicthesis
\usepackage[beramono,eulerchapternumbers,linedheaders,parts,a5paper,dottedtoc,
manychapters,pdfspacing]{classicthesis}

% Geometría y espaciado de párrafos.
\setcounter{secnumdepth}{0}
\usepackage{enumitem}
\setitemize{noitemsep,topsep=0pt,parsep=0pt,partopsep=0pt}
\setlist[enumerate]{topsep=0pt,itemsep=-1ex,partopsep=1ex,parsep=1ex}
\usepackage[top=1in, bottom=1.5in, left=1in, right=1in]{geometry}
\setlength\itemsep{0em}
\setlength{\parindent}{0pt}
\usepackage{parskip}

% Profundidad de la tabla de contenidos.
\setcounter{secnumdepth}{3}

% Usa el paquete minted para mostrar trozos de código.
% Pueden seleccionarse el lenguaje apropiado y el estilo del código.
\usepackage{minted}
\usemintedstyle{colorful}
\setminted{fontsize=\small}
\setminted[haskell]{linenos=false,fontsize=\small}
\renewcommand{\theFancyVerbLine}{\sffamily\textcolor[rgb]{0.5,0.5,1.0}{\oldstylenums{\arabic{FancyVerbLine}}}}

% Archivos de configuración.
%------------------------
% Bibliotecas para matemáticas de latex
%------------------------
\usepackage{amsmath}
\usepackage{amsthm}
\usepackage{tikz}
\usepackage{tikz-cd}
\usetikzlibrary{shapes,fit}
\usepackage{bussproofs}
\EnableBpAbbreviations{}
\usepackage{mathtools}
\usepackage{scalerel}
\usepackage{stmaryrd}

%------------------------
% Estilos para los teoremas
%------------------------
\theoremstyle{definition}
\newtheorem{definition}{Definición}[chapter]
\newtheorem*{proofs*}{Demostración}
\newtheorem{theorem}[definition]{Teorema}
\newtheorem{proposition}[definition]{Proposición}
\newtheorem{corollary}[definition]{Corolario}
\newtheorem{lemma}[definition]{Lema}
\newtheorem*{remark*}{Observación}
\newtheorem{examplex}{Ejemplo}[chapter]

\newenvironment{exampleth}
  {\pushQED{\qed}\renewcommand{\qedsymbol}{$\blacktriangle$}\examplex}
  {\popQED\endexamplex}

\setcounter{MaxMatrixCols}{20}
\newcounter{CadeanasMarkov}
\setcounter{CadeanasMarkov}{1}

\newcounter{HMM}
\setcounter{HMM}{1}
\newcommand{\HMMadd}[0]{\tag{2.\arabic{HMM}} \stepcounter{HMM}}

\begingroup\makeatletter\@for\theoremstyle:=definition,remark,plain\do{\expandafter\g@addto@macro\csname th@\theoremstyle\endcsname{\addtolength\thm@preskip\parskip}}\endgroup

%------------------------
% Macros
% ------------------------

% Aquí pueden añadirse abreviaturas para comandos de latex
% frequentemente usados.
\newcommand*\diff{\mathop{}\!\mathrm{d}}
\newcommand{\enquote}[1]{``#1''}
\DeclareMathOperator*{\argmax}{\arg\max}

\makeatletter
\DeclareRobustCommand{\rvdots}{%
  \vbox{
    \baselineskip4\p@\lineskiplimit\z@
    \kern-\p@
    \hbox{.}\hbox{.}\hbox{.}
  }}
\makeatother

\renewcommand{\emptyset}{\font\cmsy = cmsy10
 \hbox{\cmsy \char 59}
}


%---------------------Comandos para algoritmos---------------------------------------------------------------------------------
\usepackage{algorithm} % NOTE THE "plain" option HERE
\usepackage{algpseudocode}% http://ctan.org/pkg/algorithmicx
\makeatletter
\newenvironment{breakablealgorithm}
  {% \begin{breakablealgorithm}
   \begin{center}
     \refstepcounter{algorithm}% New algorithm
     \hrule height.8pt depth0pt \kern2pt% \@fs@pre for \@fs@ruled
     \renewcommand{\caption}[2][\relax]{% Make a new \caption
       {\raggedright\textbf{\fname@algorithm~\thealgorithm} ##2\par}%
       \ifx\relax##1\relax % #1 is \relax
         \addcontentsline{loa}{algorithm}{\protect\numberline{\thealgorithm}##2}%
       \else % #1 is not \relax
         \addcontentsline{loa}{algorithm}{\protect\numberline{\thealgorithm}##1}%
       \fi
       \kern2pt\hrule\kern2pt
     }
  }{% \end{breakablealgorithm}
     \kern2pt\hrule\relax% \@fs@post for \@fs@ruled
   \end{center}
  }
\makeatother
\algnewcommand{\LineComment}[1]{\State //#1}
\usepackage{etoolbox}

\makeatletter
% start with some helper code
% This is the vertical rule that is inserted
\newcommand*{\algrule}[1][\algorithmicindent]{%
  \hspace*{.2em}% <------------- This is where the rule starts from
  \vrule %height .75\baselineskip depth .25\baselineskip
  \hspace*{\dimexpr#1-.2em-.4pt}%
}

\newcommand{\StatePar}[1]{%
  \State\parbox[t]{\dimexpr\linewidth-\ALG@thistlm}{\strut #1\strut}%
}
\renewcommand{\ALG@beginalgorithmic}{\offinterlineskip}% Remove all interline skips

\newcount\ALG@printindent@tempcnta
\def\ALG@printindent{%
  \ifnum \theALG@nested > 0% is there anything to print
    \ifx\ALG@text\ALG@x@notext% is this an end group without any text?
      % do nothing
    \else
      \unskip
      % draw a rule for each indent level
      \ALG@printindent@tempcnta=1
      \loop
        \algrule[\csname ALG@ind@\the\ALG@printindent@tempcnta\endcsname]%
        \advance \ALG@printindent@tempcnta 1
        \ifnum \ALG@printindent@tempcnta<\numexpr\theALG@nested+1\relax
      \repeat
        \fi
    \fi
}
% the following line injects our new indent handling code in place of the default spacing
\patchcmd{\ALG@doentity}{\noindent\hskip\ALG@tlm}{\ALG@printindent}{}{\errmessage{failed to patch}}
% end vertical rule patch for algorithmicx
\makeatother

% Add \struts to keywords
\algrenewcommand\algorithmicend{\strut\textbf{end}}
\algrenewcommand\algorithmicdo{\strut\textbf{do}}
\algrenewcommand\algorithmicwhile{\strut\textbf{while}}
\algrenewcommand\algorithmicfor{\strut\textbf{for}}
\algrenewcommand\algorithmicforall{\strut\textbf{for all}}
\algrenewcommand\algorithmicloop{\strut\textbf{loop}}
\algrenewcommand\algorithmicrepeat{\strut\textbf{repeat}}
\algrenewcommand\algorithmicuntil{\strut\textbf{until}}
\algrenewcommand\algorithmicprocedure{\strut\textbf{procedure}}
\algrenewcommand\algorithmicfunction{\strut\textbf{function}}
\algrenewcommand\algorithmicif{\strut\textbf{if}}
\algrenewcommand\algorithmicthen{\strut\textbf{then}}
\algrenewcommand\algorithmicelse{\strut\textbf{else}}

\algrenewcommand\algorithmicrequire{\strut\textbf{Input:}}
\algrenewcommand\algorithmicensure{\strut\textbf{Output:}}

\let\oldState\State
\renewcommand{\State}{\oldState\strut}
\floatname{algorithm}{Algoritmo}
%------------------------------------------------------------------------------------------------------------------------------
  % En macros.tex se almacenan las opciones y comandos para escribir matemáticas.
\input{classicthesis-config} % En classicthesis-config.tex se almacenan las opciones propias de la plantilla.

% Color institucional UGR
% \definecolor{ugrColor}{HTML}{ed1c3e} % Versión clara.
\definecolor{ugrColor}{HTML}{c6474b}  % Usado en el título.
\definecolor{ugrColor2}{HTML}{c6474b} % Usado en las secciones.

% Datos de portada
\usepackage{titling} % Facilita los datos de la portada
\author{XuSheng Zheng} 
\date{\today}
\title{Modelos de Markov ocultos\\ y \\aplicaciones a la biología}

% Portada
\include{titlepage}
\usepackage{wallpaper}
\usepackage[main=spanish]{babel}


\begin{document}

\ThisULCornerWallPaper{1}{ugrA4.pdf}
\maketitle
\tableofcontents


\chapter*{Resumen}

% Los artículos y libros incluidos en el archivo research.bib pueden
% citarse desde cualquier punto del texto usando ~\cite.

Nos basamos en el trabajo desarrollado en~\cite{turing1936a}.

Occaecati expedita cumque est. Aut odit vel nobis praesentium dolorem
sed eligendi. Inventore molestiae delectus voluptatibus
consequatur. Et cumque quia recusandae fugiat earum repellat
porro. Earum et tempora vel voluptas. At sed animi qui hic eaque
velit.

Saepe deleniti aut voluptatem libero dolores illum iusto
iusto. Explicabo dolor quia id enim molestiae praesentium sit. Odit
enim doloribus aut assumenda recusandae. Eligendi officia nihil
itaque. Quas fugiat aliquid qui est.

Quis amet sint enim. Voluptatem optio quia voluptatem. Perspiciatis
molestiae ut laboriosam repudiandae nihil.


\ctparttext{
  \color{black}
  \begin{center}
    Esta es una descripción de la parte de matemáticas.
    Nótese que debe escribirse antes del título
  \end{center}
}
\part{Parte de matemáticas}

\chapter{Introducción a las cadenas de Markov}

En este capítulo vamos a inicializar la teoría de cadenas de Markov, avanzado progresivamente hacia las cadenas de Markov ocultas. En primer lugar, vamos a presentar el concepto de procesos de Markov:

\section{Propiedad de Markov}


Sea $\mathbb{S}$ un conjunto finito de forma $\{s_1,...,s_n\}$, definimos un proceso estocástico sobre $\mathbb{S}$ como una secuencia de variables aleatorias $\{\mathcal{X}_0,\mathcal{X}_1,\mathcal{X}_2,...\}$, o $\{\mathcal{X}_t\}_{t=0}^{\infty}$ para acortar, donde cada $\mathcal{X}_t$ es una variable aleatoria que toma valores en $\mathbb{S}$.


 A pesar de que el índice $t$ puede representar cualquiera magnitud, lo más común es que represente el tiempo. Si tomamos el índice $t$ como el tiempo, obtenemos la noción de \enquote{pasado} y \enquote{futuro}, esto es, si $t<t'$, entonces $\mathcal{X}_t$ es una variable \enquote{pasada} para $\mathcal{X}_{t'}$, mientras que $\mathcal{X}_{t'}$ es una variable \enquote{futura} para $\mathcal{X}_t$. Sin embargo, esto no es siempre así, por ejemplo, si el proceso estocástico corresponde al de las secuencias de genomas de un organismo, el conjunto $\mathbb{S}$ estará formado por los cuatro símbolos para las subunidades de nucleótidos $\{A,C,G,T\}$ y las secuenciaciones tienen un significado más espacial que temporal.
 
\begin{definition}
Un proceso estocástico $\{\mathcal{X}_t\}_{t=0}^{\infty}$ se dice que posee \textbf{la propiedad de Markov}, o es un \textbf{proceso de Markov}, si para todo $t\geq1$ y $(u_0,...,u_{t-1},u_t)\in\mathbb{S}^{t+1}$ se tiene que:
\[ \tag{1.1}\label{eq1.1}
    P[\mathcal{X}_t=u_t|\mathcal{X}_0=u_0,...,\mathcal{X}_{t-1}=u_{t-1}]=P[\mathcal{X}_t=u_t|\mathcal{X}_{t-1}=u_{t-1}]
\]
\end{definition}

La propiedad de Markov afirma que las distribuciones condicionadas del \enquote{estado actual} $\mathcal{X}_t$ depende únicamente del \enquote{pasado inmediato} $\mathcal{X}_{t-1}$ y no depende de ninguno de los anteriores estados. 

Por conveniencia, introducimos la notación $\mathcal{X}_j^k$ para denotar los estados $\mathcal{X}_i$ con $j\leq i\leq k$. Con esta notación, podemos reescribir la Definición 1.1 como sigue: Un proceso estocástico $\{\mathcal{X}_t\}$ es un \textbf{proceso de Markov} si, para todo $(u_0,...,u_{t-1},u_t)\in\mathbb{S}^{t+1}$ es cierto que:
\[ \tag{1.2}\label{eq1.2}
    P[\mathcal{X}_t=u_t|\mathcal{X}_0^{t-1}=u_0...u_{t-1}]=P[\mathcal{X}_t=u_t|\mathcal{X}_{t-1}=u_{t-1}]
\]

Para cualquier proceso estocástico $\{\mathcal{X}_t\}$ y cualquiera secuencia $(u_0,...,u_{t-1},u_t)\in\mathbb{S}^{t+1}$, tenemos que por definición de probabilidad condicionada:

\[
P[\mathcal{X}_0^t=u_0...u_t]=P[\mathcal{X}_0=u_0]\cdot\prod_{i=0}^{t-1}P[\mathcal{X}_{i+1}=u_{i+1}|\mathcal{X}_0^i=u_0...u_i]
\]

Sin embargo, si consideramos un proceso de Markov, entonces la fórmula anterior se reduce a:
\[ \tag{1.3} \label{eq1.3}
P[\mathcal{X}_0^t=u_0...u_t]=P[\mathcal{X}_0=u_0]\cdot\prod_{i=0}^{t-1}P[\mathcal{X}_{i+1}=u_{i+1}|\mathcal{X}_i=u_i]
\]

En probabilidad, es usual referirse con el nombre \textbf{cadena de Markov} a un proceso de Markov $\mathcal{X}_t$ donde el parámetro $t$ toma únicamente valores discretos. En este trabajo, pondremos nuestra atención en los casos donde $t$ toma valores en $\mathbb{Z}_+$.

En (\ref{eq1.3}) vemos la importancia del valor:
\[
P[\mathcal{X}_{t+1}=u|\mathcal{X}_t=v]
\]

al que podemos identificar como una función de tres variables: el estado \enquote{actual} $v\in\mathbb{S}$, el estado \enquote{siguiente} $u\in\mathbb{S}$ y el \enquote{tiempo actual} $t\in\mathbb{Z}_+$. Así, teniendo en cuenta que $\mathbb{S}=\{s_1,...,s_n\}$ definimos para todo tiempo $t\in\mathbb{Z}_+$:

\[ \tag{1.4} \label{eq1.4}
a_{ij}(t):=P[\mathcal{X}_{t+1}=s_j|\mathcal{X}_t=s_i]
\]

Por tanto, $a_{ij}(t)$ es la probabilidad de realizar una transición desde el estado actual $s_i$ al estado siguiente $s_j$ en el instante $t$.

\begin{definition}
Sea $\mathcal{X}_t$ una cadena de Markov, la matriz cuadrada de dimensión $n$,  $A(t)=[a_{ij}(t)]$, es la \textbf{matriz de transición} de $\mathcal{X}_t$ en el instante $t$. Una cadena de Markov es \textbf{homogénea} si $A(t)$ es constante para todo $t\in\mathbb{Z}_+$, en otro caso, es \textbf{no homogénea}. 
\end{definition}

\begin{lemma}
Sea $\mathcal{X}_t$ una cadena de Markov tomando valores en un conjunto finito $\mathbb{S}=\{s_1,...,s_n\}$, y sea $A(t)$ su matriz de transición en el instante $t$. Entonces, $A(t)$ es una \textbf{matriz estocástica} para todo $t$, esto es:
\begin{align*}
a_{ij}(t)\in[0,1]\, \forall i,j \in \{1,...,n\}, t\in\mathbb{Z}_+\\
\sum_{j=1}^n a_{ij}(t)=1 \, \forall i\in\{1,...,n\}, t\in\mathbb{Z}_+
\end{align*}
\end{lemma}

Para continuar con los estudios de las cadenas de Markov presentamos el siguiente conjunto:

\begin{definition}
El \textbf{n-símplex estándar} es el subconjunto de $\mathbb{R}^{n+1}$ dado por:
\[
\Delta^n=\{(t_1,...,t_{n+1})\in \mathbb{R}^{n+1} \, |\, \sum_{i=1}^{n+1} t_i=1 \text{ y } t_i\geq0 \text{ para todo } i\}
\]
\end{definition}

\begin{lemma}
Sea $\{\mathcal{X}_t\}$ una cadena de Markov con valores en $\mathbb{S}=\{s_1,...,s_n\}$, y sea $A(t)$ su matriz de transición en el instante $t$. Supongamos que el estado inicial $\mathcal{X}_0$ se distribuye de acuerdo con $c^0 \in \Delta^{n-1}$, esto es:
\[
P[\mathcal{X}_0=s_i]=c_i^0\, \forall i \in \{1,...n\}
\]
Entonces, para todo $t\geq0$, el estado $\mathcal{X}_t$ se distribuye de acuerdo con:
\[\tag{1.5} \label{eq1.5}
c^t=c^0A(0)A(1)...A(t-1)
\]
\end{lemma}

\part{Parte de informática}
\chapter{Sección tercera}
El siguiente código es un ejemplo de coloreado de sintaxis e inclusión
directa de código fuente en el texto usando \texttt{minted}.

\begin{minted}[frame=lines]{haskell}
-- From the GHC.Base library.
class  Functor f  where
    fmap        :: (a -> b) -> f a -> f b

    -- | Replace all locations in the input with the same value.
    -- The default definition is @'fmap' . 'const'@, but this may be
    -- overridden with a more efficient version.
    (<$)        :: a -> f b -> f a
    (<$)        =  fmap . const

-- | A variant of '<*>' with the arguments reversed.
(<**>) :: Applicative f => f a -> f (a -> b) -> f b
(<**>) = liftA2 (\a f -> f a)
-- Don't use \$ here, see the note at the top of the page

-- | Lift a function to actions.
-- This function may be used as a value for `fmap` in a `Functor` instance.
liftA :: Applicative f => (a -> b) -> f a -> f b
liftA f a = pure f <*> a
-- Caution: since this may be used for `fmap`, we can't use the obvious
-- definition of liftA = fmap.

-- | Lift a ternary function to actions.
liftA3 :: Applicative f => (a -> b -> c -> d) -> f a -> f b -> f c -> f d
liftA3 f a b c = liftA2 f a b <*> c


{-# INLINABLE liftA #-}
{-# SPECIALISE liftA :: (a1->r) -> IO a1 -> IO r #-}
{-# SPECIALISE liftA :: (a1->r) -> Maybe a1 -> Maybe r #-}
{-# INLINABLE liftA3 #-}
{-# SPECIALISE liftA3 :: (a1->a2->a3->r) -> IO a1 -> IO a2 -> IO a3 -> IO r #-}
{-# SPECIALISE liftA3 :: (a1->a2->a3->r) ->
                                Maybe a1 -> Maybe a2 -> Maybe a3 -> Maybe r #-}

-- | The 'join' function is the conventional monad join operator. It
-- is used to remove one level of monadic structure, projecting its
-- bound argument into the outer level.
join              :: (Monad m) => m (m a) -> m a
join x            =  x >>= id
\end{minted}

Vivamus fringilla egestas nulla ac lobortis. Etiam viverra est risus,
in fermentum nibh euismod quis. Vivamus suscipit arcu sed quam dictum
suscipit. Maecenas pulvinar massa pulvinar fermentum
pellentesque. Morbi eleifend nec velit ut suscipit. Nam vitae
vestibulum dui, vel mollis dolor. Integer quis nibh sapien.



% Añade sección de referencias al final del documento.
% Selecciona un estilo de cita.
\bibliographystyle{alpha}
% En research.bib están las entradas de los artículos que citamos.
% Podemos cambiar el nombre del archivo aquí.
\bibliography{research}   

\end{document}