%------------------------
% Bibliotecas para matemáticas de latex
%------------------------
\usepackage{amsmath}
\usepackage{amsthm}
\usepackage{tikz}
\usepackage{tikz-cd}
\usetikzlibrary{shapes,fit}
\usepackage{bussproofs}
\EnableBpAbbreviations{}
\usepackage{mathtools}
\usepackage{scalerel}
\usepackage{stmaryrd}

%------------------------
% Estilos para los teoremas
%------------------------
\theoremstyle{definition}
\newtheorem{definition}{Definición}[chapter]
\newtheorem*{proofs*}{Demostración}
\newtheorem{theorem}[definition]{Teorema}
\newtheorem{proposition}[definition]{Proposición}
\newtheorem{corollary}[definition]{Corolario}
\newtheorem{lemma}[definition]{Lema}
\newtheorem*{remark*}{Observación}
\newtheorem{exampleth}{Ejemplo}[chapter]
\setcounter{MaxMatrixCols}{20}
\newcounter{CadeanasMarkov}
\setcounter{CadeanasMarkov}{1}

\newcounter{HMM}
\setcounter{HMM}{1}
\newcommand{\HMMadd}[0]{\tag{2.\arabic{HMM}} \stepcounter{HMM}}

\begingroup\makeatletter\@for\theoremstyle:=definition,remark,plain\do{\expandafter\g@addto@macro\csname th@\theoremstyle\endcsname{\addtolength\thm@preskip\parskip}}\endgroup

%------------------------
% Macros
% ------------------------

% Aquí pueden añadirse abreviaturas para comandos de latex
% frequentemente usados.
\newcommand*\diff{\mathop{}\!\mathrm{d}}
\newcommand{\enquote}[1]{``#1''}
\DeclareMathOperator*{\argmax}{\arg\max}

\makeatletter
\DeclareRobustCommand{\rvdots}{%
  \vbox{
    \baselineskip4\p@\lineskiplimit\z@
    \kern-\p@
    \hbox{.}\hbox{.}\hbox{.}
  }}
\makeatother